\documentclass[12pt]{article}

\usepackage{hyperref}
\usepackage{float}
\usepackage{bussproofs}
\usepackage{minted}
\usepackage{amsmath}
\usepackage{amssymb}
\usepackage[standard,thmmarks]{ntheorem}
\usepackage{biblatex}
\usepackage{csquotes}
\usepackage{mathtools}
\usepackage{tabularx}
\usepackage[english]{babel}

\theoremstyle{break}
\newtheorem{notation}{Notation}

\addbibresource{references.bib}

\floatstyle{boxed}
\restylefloat{figure}
\allowdisplaybreaks

\begin{document}

\title{z\_ Module Specification (v1.59.0)}
\data{\today}
\author{Zakarouf \\ e-mail: \href{mailto:zak.d.official@gmail.com}{zak.d.official@gmail.com}}
\maketitle

\begin{abstract}
    This paper formally describes the syntax and semantics of various submodules of Types, Meta-Programs & Functions written in z\_. A super-sect of the C Standard Library that aims to bring user-friendly high-level abstraction and functions to C. To learn about and have a overview on z\_, please see the official repository \cite{z_}.
\end{abstract}

\tableofcontents

\newpage

\section{EBNF Grammer}
\begin{figure}[H]
    \caption{Grammer Rules}

\begin{minted}{ebnf}
    <enum> ::= "z__Enum(" <Enum-name> { "," <variant>  }+ ")" ;

    <variant> ::= "(" <variant-name> { "," <composite-type> }* ")" ;
    <Enum-name> ::= <ident> ;
    <variant-name> ::= <ident> ;

    <composite-type> ::= [ <tuple> | <record> ] ;
    <tuple> ::= <type> ;
    <record> ::= "(" <record-sub-name> { "," <type> } ")" ;
    <record-sub-name> ::= <ident> ;

    <match> ::= ["match(" | "z__Enum_match(" ] <lvalue> ")" { <arm> }+ ;
    <matches> ::= ["matches(" | "z__Enum_matches(" ] <"expr"> "," <ident> ")" ;
    <if-slot> :: ["ifSlot(" | "z__Enum_ifSlot(" ] <lvalue> "," <variant-name> ")";>
\end{minted}
\end{figure}
\end{document}
